\documentclass[twocolumn]{article}
\usepackage{amssymb}
\usepackage{amsmath}
\usepackage{proof}
\usepackage[usenames,dvipsnames]{color}

\renewcommand{\c}[1]{\ensuremath{{\color{Mahogany} {#1}}}}
\renewcommand{\v}[1]{\ensuremath{{\color{NavyBlue} {#1}}}}
\newcommand{\comp}[1]{\ensuremath{{\color{Mahogany} \heartsuit {#1}}}}
\newcommand{\carr}[2]{\ensuremath{{\color{Mahogany} {#1} \Rightarrow {#2}}}}
\newcommand{\valu}[1]{\ensuremath{{\color{NavyBlue} \spadesuit {#1}}}}
\newcommand{\varr}[2]{\ensuremath{{\color{NavyBlue} {#1} \supset {#2}}}}
\newcommand{\atseq}[3]{\ensuremath{{\color{Mahogany} {#1}};
                                  {\color{NavyBlue} {#2}}
                                  \Vdash 
                                  {\color{NavyBlue} {#3}}}\mathstrut}
\newcommand{\avseq}[2]{\ensuremath{{\color{Mahogany} {#1}}
                                  \Vdash 
                                  {\color{Mahogany} {#2}}}\mathstrut}
\begin{document}
\title{Canonical forms = datatypes = abstract binding trees}
\author{Robert J. Simmons}
\maketitle

\begin{abstract}
\end{abstract}

\section{Adjoint logic}

We begin with an presentation of Reed's {\it adjoint logic}
\cite{reed_judgmental}. The key idea of adjoint 
logic is to modify Martin L\"of's {\it judgmental methadology} by ensuring
that propositions which are judged differently belong to different syntactic
classes \cite{lof_meaning}. 
In the setting of Pfenning and Davies' judgmental reconstruction 
of S4 modal logic \cite{pfenning_judgmental}, this means separating those
propositions $\v{A}$ judged as {\it true} (which are collected in a context 
$\v{\Gamma}$) and the propositions $\c{P}$ judged as 
{\it valid} (which are collected in a context $\c{\Delta}$).
\[
\begin{array}{rrcl}
\mbox{\it Judged as valid}
 & \c{P}, \c{Q} & ::= & \valu{A} \mid \ldots\\
\mbox{\it Judged as true}
 & \v{A}, \v{B}, \v{C} & ::= & \comp{P} \mid \varr{A}{B} \mid \v{a}\\
\mbox{\it Valid contexts}
 & \c{\Delta} & ::= & \c{\cdot} \mid \c{\Delta, P} \\
\mbox{\it True contexts}
 & \v{\Gamma} & ::= & \v{\cdot} \mid \v{\Gamma, A} 
\end{array}
\]
The sequent calculus presentation of adjoint logic is given in 
Figure~\ref{fig:adjoint}.

\begin{figure}[ht]
\fbox{$\avseq{\Delta}{P}$}
\fbox{$\atseq{\Delta}{\Gamma}{A}$}
\[
\infer[\it init_{valid}]
{\avseq{\Delta, P}{P}}
{\c{P} \neq \valu{A} \mathstrut}
\qquad
\infer[\it init_{true}]
{\atseq{\Delta}{\Gamma, a}{a}}
{}
\]\[
\infer[\spadesuit_R]
{\avseq{\Delta}{\valu{A}}}
{\atseq{\Delta}{\cdot}{A}}
\qquad
\infer[\spadesuit_L]
{\atseq{\Delta, \valu{A}}{\Gamma}{C}}
{\atseq{\Delta, \valu{A}}{\Gamma, A}{C}}
\]\[
\infer[\heartsuit_R]
{\atseq{\Delta}{\Gamma}{\comp{P}}}
{\avseq{\Delta}{P}}
\qquad
\infer[\heartsuit_L]
{\atseq{\Delta}{\Gamma, \comp{P}}{C}}
{\atseq{\Delta, P}{\Gamma, \comp{P}}{C}}
\]\[
\infer[\supset_R]
{\atseq{\Delta}{\Gamma}{\varr{A}{B}}}
{\atseq{\Delta}{\Gamma, A}{B}}
\]\[\infer[\supset_L]
{\atseq{\Delta}{\Gamma, A \supset B}{C}}
{\atseq{\Delta}{\Gamma, A \supset B}{A} 
 &
 \atseq{\Delta}{\Gamma, A \supset B, B}{C}}
\]
\caption{Reed's sequent calculus for adjoint logic.}
\label{fig:adjoint}
\end{figure}


The connection with the standard judgmental presentation of S4 
modal logic can be made by declaring that 
$\v{\Box A} \equiv \comp{\valu{A}}$. This formulation just makes
explicit the long-recognized observation that the 
introduction rule for modal necessity in Pfenning-Davies S4 decomposes
into two rules, one non-invertable rule that clears the context 
(this corresponds to $\heartsuit_R$) and one invertible rule that
transitions from validity to truth. Below, we show the Pfenning-Davies
decomposition of $\Box_R$ (left) and the adjoint logic view (right). 

\[
\infer
{\Delta; \Gamma \vdash \Box A\,\mathit{true} \mathstrut}
{\infer
 {\Delta \vdash A\,\mathit{valid}\mathstrut}
 {\Delta; \cdot \vdash A\,\mathit{true}}\mathstrut}
\quad
\infer[\heartsuit_R]
{\atseq{\Delta}{\Gamma}{\comp{\valu{A}}}}
{\infer[\spadesuit_R]
 {\avseq{\Delta}{\valu{A}}}
 {\atseq{\Delta}{\Gamma}{A}}} 
\]



The interesting difference between our presentation and
Reed's is that we leave (for now) the language of propositions that 
may be judged as valid entirely open-ended; the $\mathit{init_{valid}}$
rule ensures that all such propositions will be effectively treated like
atomic propositions from the perspective of the sequent calculus.


The modalities  
$\comp{P}$ and $\valu{A}$ used to mark the 
inclusion of valid and true propositions into each other are admittedly
hideous, but they avoid colliding with existing notation. Reed makes
the adjoint nature of the modalities explicit, writing $\comp{P}$ as
$\mathit{FP}$ and $\valu{A}$ as $\mathit{UA}$. The
reason we do not use this notation is because we will eventually use 
the types of adjoint logic as scaffolding for a language of 

, in our eventual computational
language, the call-by-push-value language's 
$\mathit{U}$ very nearly corresponds with $\comp{P}$ and the 
call-by-push-value language's $\mathit{F}$ very nearly corresponds with
$\valu{A}$ -- exactly the oppositite association!




\end{document}